\documentclass[sigconf]{acmart}

\usepackage{booktabs} % For formal tables
\usepackage{amsmath}
\usepackage{algorithm}
\usepackage[noend]{algpseudocode}
\usepackage[T1]{fontenc}
\usepackage{tabularx}
\usepackage{array}
%\usepackage{txfonts}
\usepackage{pbox}

\newcolumntype{L}[1]{>{\raggedright\let\newline\\\arraybackslash\hspace{0pt}}m{#1}}
\newcolumntype{C}[1]{>{\centering\let\newline\\\arraybackslash\hspace{0pt}}m{#1}}
\newcolumntype{R}[1]{>{\raggedleft\let\newline\\\arraybackslash\hspace{0pt}}m{#1}}


% Copyright
%\setcopyright{none}
%\setcopyright{acmcopyright}
%\setcopyright{acmlicensed}
\setcopyright{rightsretained}
%\setcopyright{usgov}
%\setcopyright{usgovmixed}
%\setcopyright{cagov}
%\setcopyright{cagovmixed}


% DOI
\acmDOI{10.475/123_4}

% ISBN
\acmISBN{123-4567-24-567/08/06}

%Conference
\acmConference[GECCO '18]{the Genetic and Evolutionary Computation Conference 2018}{July 15--19, 2018}{Kyoto, Japan}
\acmYear{2018}
\copyrightyear{2018}

\acmPrice{15.00}


\begin{document}
\title{Comparative Mixing for DSMGA-II}

\author{Marcin M. Komarnicki}
\affiliation{%
  \institution{Department of Computational Intelligence}
  \institution{Wroclaw University of Science and Technology}
  \city{Wroclaw}
  \country{Poland} 
}
\email{marcin.komarnicki@pwr.edu.pl}

\author{Michal W. Przewozniczek}
\affiliation{%
	\institution{Department of Computational Intelligence}
	\institution{Wroclaw University of Science and Technology}
	\city{Wroclaw}
	\country{Poland} 
}
\email{michal.przewozniczek@pwr.edu.pl}


\begin{abstract}
	
Dependency Structure Matrix Genetic Algorithm-II (DSMGA-II) is a recently proposed optimization method that builds the linkage model on the base of Dependency Structure Matrix (DSM). This model is used during the Optimal Mixing (OM) operators such as the Restricted Mixing (RM) and the Back Mixing (BM). DSMGA-II was shown to solve theoretical and real-world optimization problems effectively. In this paper, we show that the effectiveness of DSMGA-II and its improved version, namely Two-edge Dependency Structure Matrix Genetic Algorithm-II (DSMGA-IIe), is relatively low for overlapping optimization problems. Thus, we propose the Comparative Mixing (CM) operator that extends the RM operator. The CM operator modifies the linkage information obtained from DSM-based linkage model by comparing the mixed individual with other, randomly selected member of the population. Such modification enables DSMGA-II to solve overlapping problems effectively and does not limit DSMGA-II performance on other problems for which it was already shown effective.

\end{abstract}

%
% The code below should be generated by the tool at
% http://dl.acm.org/ccs.cfm
% Please copy and paste the code instead of the example below. 
%
 \begin{CCSXML}
	<ccs2012>
		<concept>
			<concept_id>10010147.10010178</concept_id>
			<concept_desc>Computing methodologies~Artificial intelligence</concept_desc>
			<concept_significance>500</concept_significance>
		</concept>
	</ccs2012>
\end{CCSXML}

\ccsdesc[500]{Computing methodologies~Artificial intelligence}

% We no longer use \terms command
%\terms{Theory}

\keywords{\textcolor{red}{Linkage Learning, Genetic Algorithms}}


\maketitle

\section{Introduction}

Introduction

\begin{acks}
	Acks
\end{acks}


\bibliographystyle{ACM-Reference-Format}
\bibliography{sigproc} 

\end{document}
